\documentclass[%
 reprint,
%superscriptaddress,
%groupedaddress,
%unsortedaddress,
%runinaddress,
%frontmatterverbose, 
%preprint,
%preprintnumbers,
%nofootinbib,
%nobibnotes,
%bibnotes,
 amsmath,amssymb,
 aps,
%pra,
%prb,
%rmp,
%prstab,
%prstper,
%floatfix,
]{revtex4-2}

\usepackage{graphicx}
\usepackage{dcolumn}
\usepackage{bm}

\begin{document}

\preprint{APS/123-QED}

\title{Relational Knowledge Engineering Platform \\ Product Requirement Document}
\thanks{Product Requirement Document}

\author{Yongyuth Chuankhuntod}
 \email{yongyuth.chuankhuntod@email.kmutnb.ac.th}
\author{Anirach Mingkhwan}
 \email{anirach.m@fitm.kmunt.b.ac.thh}
\affiliation{Faculty of Industrial and Technology Management\\King Mongkut's University of Technology North Bangkok}

\begin{abstract}
This document serves as a Product Requirement Document (PRD) for a bachelor's thesis project, not for submission to a conference or academic journal. It outlines the requirements and specifications for developing a Relational Knowledge Engineering Platform - a web-based tool designed to extract and visualize relationships between pieces of knowledge from uploaded documents.

\begin{description}
\item[Purpose]
Bachelor's thesis project documentation and development guidance.
\item[Scope]
Complete product requirements for a knowledge engineering platform including document processing, relationship extraction, network visualization, and LLM-powered chat interface.
\end{description}
\end{abstract}

\maketitle

\section{\label{sec:introduction}Introduction}

The Relational Knowledge Engineering Platform is a web-based tool designed to help users extract and visualize relationships between pieces of knowledge contained in uploaded documents (e.g., PDFs). The platform processes text from these documents, generates a network graph to represent relationships, and provides interactive features for users to explore and manipulate the graph. Additionally, a chat interface powered by a Large Language Model (LLM) allows users to query their data, making it a powerful tool for knowledge discovery and management.

\section{\label{sec:objective}Objective}

The platform aims to:
\begin{itemize}
\item Enable users to upload text-containing files (e.g., PDFs) and automatically extract knowledge relationships.
\item Present extracted relationships as an interactive network graph.
\item Allow users to customize graph generation, interact with the graph (extract, add, delete elements), and summarize it based on metadata like text sentences in nodes.
\item Provide a chat interface with an LLM for querying the graph and documents.
\item Offer a user-friendly experience with account management and an engaging landing page.
\end{itemize}

\section{\label{sec:target_audience}Target Audience}

\begin{itemize}
\item Researchers, analysts, and students needing to analyze relationships within textual data.
\item Professionals in knowledge management, data science, or education seeking visual and interactive tools for text analysis.
\end{itemize}

\section{\label{sec:key_features}Key Features}

\subsection{\label{sec:landing_page}Landing Page}
\begin{description}
\item[Purpose] Introduce the platform and attract users.
\item[Features] ~
\begin{itemize}
\item Overview of the platform's capabilities (e.g., knowledge extraction, graph visualization, LLM chat).
\item Call-to-action buttons for signing up or logging in.
\item Examples or use cases highlighting the platform's value.
\end{itemize}
\end{description}

\subsection{\label{sec:account_management}Account Management (Own Account)}
\begin{description}
\item[Purpose] Allow users to manage their personal settings.
\item[Features] ~
\begin{itemize}
\item Profile management (e.g., name, email).
\item Password reset or change functionality.
\item User preferences (e.g., notification settings).
\end{itemize}
\end{description}

\vspace{1cm}

\subsection{\label{sec:graph_viewer}Graph Viewer}
\begin{description}
\item[Purpose] Display and interact with the network graph.
\item[Features] ~
\begin{itemize}
\item Interactive graph visualization with zoom, pan, and node/edge selection.
\item Display of node details (e.g., text sentences) and edge relationships.
\item Tools to:
\begin{itemize}
\item \textbf{Extract}: Highlight or isolate specific nodes/edges.
\item \textbf{Add}: Manually insert new nodes or relationships.
\item \textbf{Delete}: Remove nodes or edges.
\end{itemize}
\item Summarize the graph based on metadata (e.g., key sentences or themes).
\end{itemize}
\end{description}

\subsection{\label{sec:chat_llm}Chat with LLM}
\begin{description}
\item[Purpose] Enable users to query their data conversationally.
\item[Features] ~
\begin{itemize}
\item Chat interface for inputting questions about the graph or documents.
\item LLM responses tailored to the user's uploaded content.
\item Optional integration with the graph viewer (e.g., referencing specific nodes in responses).
\end{itemize}
\end{description}

\subsection{\label{sec:file_upload}File Upload and Graph Customization}
\begin{description}
\item[Purpose] Handle document uploads and graph generation settings.
\item[Features] ~
\begin{itemize}
\item Upload support for files like PDFs.
\item Text extraction and relationship identification from uploaded documents.
\item Customization options (e.g., select document sections, adjust relationship extraction parameters).
\item Preview of the generated graph before finalizing.
\end{itemize}
\end{description}

\section{\label{sec:technical_requirements}Technical Requirements}

\subsection{\label{sec:frontend}Frontend}
\begin{itemize}
\item \textbf{Framework}: Next.ts (TypeScript) for a dynamic, interactive interface.
\item \textbf{Graph Visualization}: Reagraph for rendering and interacting with the network graph.
\item \textbf{Design}: Responsive layout compatible with desktop and mobile browsers.
\end{itemize}

\subsection{\label{sec:backend}Backend}
\begin{itemize}
\item \textbf{Framework}: Go or Python for robust server-side logic.
\item \textbf{NLP}: spaCy or transformers for entity and relationship extraction.
\end{itemize}

\subsection{\label{sec:database}Database}
\begin{itemize}
\item \textbf{Graph Database}: Neo4j graph database for storing and querying knowledge relationships.
\item \textbf{Vector Database}: Qdrant for storing and querying document embeddings and context for LLM memory.
\item \textbf{General Database}: MongoDB for general-purpose data storage (e.g., user accounts, settings, metadata).
\item \textbf{Cache Database}: DragonflyDB for caching and pub/sub messaging.
\item \textbf{Metadata}: Store node/edge details (e.g., text sentences) alongside graph structure.
\end{itemize}

\subsection{\label{sec:file_storage}File Storage}
\begin{itemize}
\item \textbf{Object Storage}: MinIO for storing all uploaded files (PDFs, documents) with S3-compatible API.
\item \textbf{File Management}: Secure file upload, storage, and retrieval with proper access controls.
\item \textbf{Backup}: Automated backup and versioning of stored files.
\end{itemize}

\subsection{\label{sec:llm_integration}LLM Integration}
\begin{itemize}
\item \textbf{Provider}: API integration with OpenAI or Hugging Face for LLM functionality.
\item \textbf{Context}: Pass graph data and document text to the LLM for contextual responses.
\end{itemize}

\subsection{\label{sec:security}Security}
\begin{itemize}
\item \textbf{Authentication}: JWT-based user authentication.
\item \textbf{Data Protection}: HTTPS and encryption for uploaded files and user data.
\end{itemize}

\subsection{\label{sec:scalability}Scalability}
\begin{itemize}
\item \textbf{Infrastructure}: Cloud hosting (e.g., XVER) to support multiple users and large files.
\item \textbf{Optimization}: Efficient graph generation and querying for performance.
\end{itemize}

\section{\label{sec:user_workflow}User Workflow}

\begin{enumerate}
\item \textbf{Sign Up/Log In}: Users create an account or log in via the landing page.
\item \textbf{Upload File}: Users upload a PDF on the file upload page.
\item \textbf{Customize Graph}: Users adjust settings for relationship extraction and preview the graph.
\item \textbf{View Graph}: Users explore the network graph in the graph viewer, interacting with nodes and edges.
\item \textbf{Modify Graph}: Users extract, add, or delete elements and summarize the graph as needed.
\item \textbf{Chat with LLM}: Users ask questions about their data via the chat interface.
\item \textbf{Manage Account}: Users update their profile or settings on the account management page.
\end{enumerate}

\section{\label{sec:success_criteria}Success Criteria}

\begin{itemize}
\item \textbf{Functionality}: Accurate graph generation from uploaded files with full interactivity (extract/add/delete/summarize).
\item \textbf{Usability}: Intuitive navigation and clear interfaces across all pages.
\item \textbf{Performance}: Fast processing of documents and responsive graph interaction.
\item \textbf{Insightfulness}: LLM provides relevant, accurate answers based on user data.
\end{itemize}

\section{\label{sec:future_considerations}Future Considerations}

\begin{itemize}
\item Support for additional file formats (e.g., DOCX, TXT).
\item Advanced graph analytics (e.g., clustering, centrality measures).
\item Multi-user collaboration on shared graphs.
\end{itemize}

\subsection{\label{sec:level2}Second-level heading: Formatting}

\subsection{\label{sec:document_structure}Document Structure}

This document provides a comprehensive Product Requirement Document (PRD) for the Relational Knowledge Engineering Platform, structured to guide the development process from initial concept to implementation.

\begin{acknowledgments}
This project is developed as part of a bachelor's thesis at King Mongkut's University of Technology North Bangkok, Faculty of Industrial and Technology Management.
\end{acknowledgments}

\end{document}
